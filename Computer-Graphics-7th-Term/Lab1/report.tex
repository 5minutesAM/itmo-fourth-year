\documentclass[listings]{labreport}
\subject{Компьютерная графика}
\titleparts{Лабораторная работа №1}{Вариант 8}
\students{Лабушев Тимофей}

\usepackage{enumitem}

\begin{document}

\maketitlepage

\section*{Задание}

\subsection*{DOS}

В среде DOSBox с использованием 16-битного набора команд x86 продемонстрировать
взаимодействие с видеоадаптером VGA в режиме \verb|12h| (разрешение 640x480, 16 цветов).

Разработать процедуры:
\begin{enumerate}[noitemsep,topsep=0em]
\item вывода на экран точки с заданными координатами;
\item построения горизонтальной линии произвольного размера (от 1 до 640 пикселей);
\item построения вертикальной линии произвольного размера (от 1 до 480 пикселей);
\item построения фигуры с заливкой, используя алгоритм с затравкой.
\end{enumerate}
\vspace{0.4em}

В качестве фигуры для построения взять первую букву фамилии.

\subsection*{GDI}

Реализовать построение фигуры с заливкой аналогично последнему пункту задания для DOS.

\section*{Описание реализации}

\subsection*{DOS}

Для построения фигуры принято решение использовать кубические кривые Безье.
Основной мотивацией является их применение в векторной графике, в частности, формате SVG,
что позволяет составить эталонную фигуру в графическом редакторе.

Преобразование SVG файла с элементами \verb|<path>| в формат, заносимый в
ассемблерный листинг, производится автоматически с помощью специально написанного скрипта
\verb|svg_to_curves|.

Построение кривой выполняется итеративно на основе следующего уравнения:
\[
{\mathbf{P}}(t)=(1-t)^{3}{\mathbf{P}}_{0}+3t(1-t)^{2}{\mathbf{P}}_{1}+3t^{2}(1-t){\mathbf{P}}_{2}+t^{3}{\mathbf{P}}_{3},\quad t\in [0,1]
\]

Шаг изменения $t$ принят равным $0.01$. Данное значение обеспечивает достаточную точность
построения без видимых для пользователя задержек (при конфигурации \verb|cycles=3000| в DOSBox).

Полученные точки $\mathbf{P}$ соединяются прямыми линиями, рисуемыми при помощи\\
алгоритма Брезенхэма.

Заливка фигуры производится построчным алгоритмом закраски с затравкой.

\subsection*{GDI}

Использованы библиотечные функции \verb|PolyBezier| (построение кривых Безье) и\\
\verb|ExtFloodFill| (заливка).

\section*{Скриншоты реализации}

\subsection*{DOS}

\begin{figure}[h!]
\centering
\includegraphics[width=0.6\textwidth]{bezier1.png}
\caption{Построенная фигура (первая буква фамилии) без заливки}
\end{figure}

\begin{figure}[h!]
\centering
\includegraphics[width=0.6\textwidth]{bezier2.png}
\caption{Построенная фигура с заливкой}
\end{figure}

\newpage
\subsection*{GDI}

\begin{figure}[h!]
\centering
\includegraphics[width=0.6\textwidth]{bezier_gdi1.jpg}
\caption{Построенная фигура (первая буква фамилии) без заливки}
\end{figure}

\begin{figure}[h!]
\centering
\includegraphics[width=0.6\textwidth]{bezier_gdi2.jpg}
\caption{Построенная фигура с заливкой}
\end{figure}

\section*{Исходный код}

Исходный код лабораторной работы доступен по адресу:

\verb|https://github.com/timlathy/itmo-fourth-year/tree/master/Computer-Graphics-7th-Term/Lab1|.

\end{document}
